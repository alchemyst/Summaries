\documentclass[float=false, crop=false]{standalone}
\linespread{1.3}

\usepackage[subpreambles=true]{standalone}
\usepackage{import}

\usepackage{parskip}
\setlength{\parindent}{0pt} %no paragraph indentation
\setlength{\parskip}{2.1ex plus 0.2ex minus 0.2ex} %3x paragraph spacing

\usepackage{geometry}
\geometry{letterpaper,left=1.0in,right=1.0in,top=1.0in,bottom=1.0in}

\usepackage{fancyhdr}
\pagestyle{fancy}
\fancyhf{}
\renewcommand{\headrulewidth}{0pt}
\rfoot{\thepage}

\usepackage{hyperref}

\usepackage[fleqn]{amsmath}
\usepackage{amssymb}
\usepackage{amsthm}

\usepackage{graphicx}
\graphicspath{{./imgs/}}
\usepackage{float}

\begin{document}
	In \cite{Wyatt2016} the authors discuss the evolutionary benefits interacting species may obtain if both restrict access to nutrients important to their mutual trade partner. The authors develop a fitness model describing the interaction of plants and mycorrhizal fungi. Plants evolved to specialize in carbon uptake while the fungi specialize in phosphorous uptake. The fungi restricts the plant's ability to absorb phosphorous which makes the phosphorous a valuable trading commodity. This improves the terms of trade for the two commodities (since the fungi are poor at carbon sequestration) and effectively makes cooperation beneficial, even in environments where cooperative behavior might have collapsed otherwise e.g. high nutrient soil. Essentially each species has a comparative advantage over the other which promotes mutualistic trade.  	

	\ifstandalone
		\bibliographystyle{unsrt}	
		\bibliography{../../../Research/library}
	\fi

\end{document}