\documentclass[float=false, crop=false]{standalone}
\linespread{1.3}

\usepackage[subpreambles=true]{standalone}
\usepackage{import}

\usepackage{parskip}
\setlength{\parindent}{0pt} %no paragraph indentation
\setlength{\parskip}{2.1ex plus 0.2ex minus 0.2ex} %3x paragraph spacing

\usepackage{geometry}
\geometry{letterpaper,left=1.0in,right=1.0in,top=1.0in,bottom=1.0in}

\usepackage{fancyhdr}
\pagestyle{fancy}
\fancyhf{}
\renewcommand{\headrulewidth}{0pt}
\rfoot{\thepage}

\usepackage{hyperref}

\usepackage[fleqn]{amsmath}
\usepackage{amssymb}
\usepackage{amsthm}

\usepackage{graphicx}
\graphicspath{{./imgs/}}
\usepackage{float}

\begin{document}
	RNA-seq is a modern tool for transcriptome profiling that uses next generation sequencing \cite{Wang2009}. Essentially a population of RNA is converted into a library of cDNA with adaptors attached at each end. Each resultant molecule is then sequenced and hence the transcriptome can be inferred. The benefits of this approach include: a high degree of reproducibility and throughput. However, there are some challenges with this new technology. Long RNA/cDNA molecules need to be fragmented which can make library reconstruction difficult.  

	\ifstandalone
		\bibliographystyle{unsrt}	
		\bibliography{../../../Research/library}
	\fi

\end{document}