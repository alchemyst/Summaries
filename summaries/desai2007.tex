\documentclass[float=false, crop=false]{standalone}
\linespread{1.3}

\usepackage[subpreambles=true]{standalone}
\usepackage{import}

\usepackage{parskip}
\setlength{\parindent}{0pt} %no paragraph indentation
\setlength{\parskip}{2.1ex plus 0.2ex minus 0.2ex} %3x paragraph spacing

\usepackage{geometry}
\geometry{letterpaper,left=1.0in,right=1.0in,top=1.0in,bottom=1.0in}

\usepackage{fancyhdr}
\pagestyle{fancy}
\fancyhf{}
\renewcommand{\headrulewidth}{0pt}
\rfoot{\thepage}

\usepackage{hyperref}

\usepackage[fleqn]{amsmath}
\usepackage{amssymb}
\usepackage{amsthm}

\usepackage{graphicx}
\graphicspath{{./imgs/}}
\usepackage{float}

\begin{document}
	
	In \cite{Desai2007} the authors investigate the rate at which beneficial mutations accumulate in asexual populations. Traditionally it is believed that so-called one-by-one clonal interference is the dominating dynamic. In this regime the overall speed of evolution is limited because the dominant mutation needs to fix itself before the mutation process can continue. Thus beneficial mutations could be lost if they do not occur in the group of cells which eventually become fixed. The authors extend this model by accounting for multiple mutations which may occur in a population before the most fit mutation becomes fixed. This broadens the fitness distribution of the resultant population and can lead to a situation where an initially less fit mutation eventually fixes itself in the population. This happens because the less fit mutation has the opportunity to mutate again and thus increase its fitness. Experimentally the authors found that the rate of evolution is dominated by the accumulation of multiple mutations. The multiple mutation hypothesis predicts that the speed of evolution increase logarithmically with total population size and mutation rate.  

	\ifstandalone
		\bibliographystyle{unsrt}	
		\bibliography{../../../Research/library}
	\fi

\end{document}