\documentclass[float=false, crop=false]{standalone}
\linespread{1.3}

\usepackage[subpreambles=true]{standalone}
\usepackage{import}

\usepackage{parskip}
\setlength{\parindent}{0pt} %no paragraph indentation
\setlength{\parskip}{2.1ex plus 0.2ex minus 0.2ex} %3x paragraph spacing

\usepackage{geometry}
\geometry{letterpaper,left=1.0in,right=1.0in,top=1.0in,bottom=1.0in}

\usepackage{fancyhdr}
\pagestyle{fancy}
\fancyhf{}
\renewcommand{\headrulewidth}{0pt}
\rfoot{\thepage}

\usepackage{hyperref}

\usepackage[fleqn]{amsmath}
\usepackage{amssymb}
\usepackage{amsthm}

\usepackage{graphicx}
\graphicspath{{./imgs/}}
\usepackage{float}

\begin{document}
	In \cite{VerBerkmoes2009} the authors describe how proteomics can be used to characterize microbial consortia. The authors define metaproteomics as the study of complex systems where the exact origin of the proteins present cannot be established; thus only the metabolic activity of the community can be studied. By contrast, proteomics and genomics are the study of systems where the microbe responsible for the protein can be ascertained. Furthermore, the authors differentiate between top down and bottom up proteomics. Top down proteomics is where whole proteins are separated using liquid chromatography and then analyzed using mass spectrometry. Bottom up proteomics is when the proteins are lysed prior to identification using mass spectrometry. The authors assert that label free methods are becoming a popular method for proteom quantification. 

	\ifstandalone
		\bibliographystyle{unsrt}	
		\bibliography{../../../Research/library}
	\fi

\end{document}