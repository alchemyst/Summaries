\documentclass[float=false, crop=false]{standalone}
\linespread{1.3}

\usepackage[subpreambles=true]{standalone}
\usepackage{import}

\usepackage{parskip}
\setlength{\parindent}{0pt} %no paragraph indentation
\setlength{\parskip}{2.1ex plus 0.2ex minus 0.2ex} %3x paragraph spacing

\usepackage{geometry}
\geometry{letterpaper,left=1.0in,right=1.0in,top=1.0in,bottom=1.0in}

\usepackage{fancyhdr}
\pagestyle{fancy}
\fancyhf{}
\renewcommand{\headrulewidth}{0pt}
\rfoot{\thepage}

\usepackage{hyperref}

\usepackage[fleqn]{amsmath}
\usepackage{amssymb}
\usepackage{amsthm}

\usepackage{graphicx}
\graphicspath{{./imgs/}}
\usepackage{float}

\begin{document}
	In \cite{Choudhary2010} the authors primarily discuss recent advances in quorum sensing technology. It is noted that the most commonly used quorum sensing synthetic systems reported in literature are those which use the AHL sensing mechanism most often used by Gram-negative bacteria. Specifically the \textit{LuxI}/\textit{LuxR} genes from Vibrio fischeri and the \textit{LasI}/\textit{LasR}, \textit{RhlI}/\textit{RhlR} genes from Pseudomonas aeruginosa. While these communication modules are often used care must be taken since it has been found that the products of those genes sometimes elicit unwanted responses in engineered systems e.g. immune responses. A drawback with using AHL systems is that there is usually a fair amount of crosstalk between different signals and receptors. The literature reports that the AIP systems used by Gram-positive bacteria are extremely specific and does not have the associated noise problem. However, these sensing modules have a higher metabolic cost associated with them.

	\ifstandalone
		\bibliographystyle{unsrt}	
		\bibliography{../../../Research/library}
	\fi

\end{document}