\documentclass[float=false, crop=false]{standalone}
\linespread{1.3}

\usepackage[subpreambles=true]{standalone}
\usepackage{import}

\usepackage{parskip}
\setlength{\parindent}{0pt} %no paragraph indentation
\setlength{\parskip}{2.1ex plus 0.2ex minus 0.2ex} %3x paragraph spacing

\usepackage{geometry}
\geometry{letterpaper,left=1.0in,right=1.0in,top=1.0in,bottom=1.0in}

\usepackage{fancyhdr}
\pagestyle{fancy}
\fancyhf{}
\renewcommand{\headrulewidth}{0pt}
\rfoot{\thepage}

\usepackage{hyperref}

\usepackage[fleqn]{amsmath}
\usepackage{amssymb}
\usepackage{amsthm}

\usepackage{graphicx}
\graphicspath{{./imgs/}}
\usepackage{float}

\begin{document}
	In \cite{Dwork2015} the authors derive an algorithm, called Thresholdout, which aims to preserve the statistical validity of inferences drawn from a data set even if the data is used multiple times. Specifically, the authors provide a bound on the number of times a validation data set may be reused before it loses statistical significance and becomes prone to over-fitting. 
	
	\ifstandalone
		\bibliographystyle{unsrt}	
		\bibliography{../../../Research/library}
	\fi

\end{document}